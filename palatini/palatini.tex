\documentclass[portuguese]{artigo}

\title{Variação de Palatini}
\author{Louis Bergamo Radial\\8992822}

\begin{document}
\maketitle
\begin{lemma}{A diferença entre conexões é um tensor}{conexões_tensor}
        Seja \((M, g)\) uma variedade Lorentziana. Se \(\nabla, \tilde{\nabla} : \sections{TM} \times \sections{TM} \to \sections{TM}\) são conexões afins, então a aplicação
        \begin{align*}
            D : \sections{TM} \times \sections{TM} &\to \sections{TM}\\
                                             (X,Y) &\mapsto \nabla_XY - \tilde{\nabla}_XY
        \end{align*}
        é um campo tensorial de ordem \((1, 2)\). Ainda, se os coeficientes das conexões em uma dada carta local são \(\Gamma\indices{^k_{ij}}\) e \(\tilde{\Gamma}\indices{^k_{ij}}\), então \(D\indices{^k_{ij}} = \Gamma\indices{^k_{ij}} - \tilde{\Gamma}\indices{^k_{ij}}\) é a expressão em coordenadas locais deste tensor.
    \end{lemma}
    \begin{proof}
        Consideremos \(f,g\in\smooth{M}\) e \(X,Y,Z \in\sections{TM}\) e mostremos que \(D\) é linear na dependência nos dois campos vetoriais. Utilizando a linearidade no primeiro argumento de uma conexão, segue que
        \begin{align*}
            D(fX + gY, Z) &= \nabla_{fX + gY}Z - \tilde{\nabla}_{fX + gY}Z\\
                          &= f \nabla_X Z + g\nabla_Y Z - f\nabla_X Z - g\nabla_Y Z\\
                          &= f D(X,Z) + g D(Y,Z),
        \end{align*}
        logo \(D\) é linear em seu primeiro argumento. Pela aditividade e pela regra de Leibniz, obtemos
        \begin{align*}
            D(X, fY + gZ) &= \nabla_X (fY) + \nabla_X (gZ) - \tilde{\nabla}_X (fY) - \tilde{\nabla}_X (gZ)\\
                          &= \left[(Xf)Y + f\nabla_X Y\right] + \left[(Xg)Z + g\nabla_X Z\right] - \left[(Xf)Y + f\tilde{\nabla}_X Y\right] - \left[(Xg)Z + g\tilde{\nabla}_X Z\right]\\
                          &=  fD(X,Y) + gD(X,Z),
        \end{align*}
        portanto \(D\) é linear em seu segundo argumento. Por fim, notando que \(D(X,Y)\) é a combinação linear de dois campos vetoriais, segue que \(D(X,Y) \in \sections{TM}\) para todo \(X,Y\in \sections{TM}\), portanto concluímos que \(D\) é um tensor de ordem \((1,2)\).

        Seja \((U, x) \in \mathscr{A}_M\) uma carta de coordenadas locais, então
        \begin{align*}
            D\indices{^k_{ij}} &= \inner*{\dl{x}^k}{D\left(\bfield{x^i}, \bfield{x^j}\right)}\\
                               &= \inner*{\dl{x}^k}{\nabla_{\bfield{x^i}}\bfield{x^j} - \tilde{\nabla}_{\bfield{x^i}}\bfield{x^j}}\\
                               &= \inner*{\dl{x}^k}{\nabla_{\bfield{x^i}}\bfield{x^j}} - \inner*{\dl{x}^k}{\tilde{\nabla}_{\bfield{x^i}}\bfield{x^j}}\\
                               &= \Gamma\indices{^k_{ij}} - \tilde{\Gamma}\indices{^k_{ij}},
        \end{align*}
        como desejado.
\end{proof}

\begin{proposition}{Expressão em coordenadas locais do tensor de curvatura}{Riemann}
    Seja \((M, g)\) uma variedade Lorentziana dotada de uma conexão \(\nabla\). O tensor de curvatura de Riemann, definido pela aplicação
    \begin{align*}
        R : \sections{TM} \times \sections{TM} \times \sections{TM} &\to \sections{TM}\\
        (X, Y, Z) &\mapsto R(X,Y)Z = \nabla_X \nabla_Y Z - \nabla_Y \nabla_X Z - \nabla_{[X,Y]}Z,
    \end{align*}
    tem componentes em coordenadas locais dadas por
    \begin{equation*}
        R\indices{^\ell_{kij}} = \partial_i \Gamma\indices{^\ell_{jk}} - \partial_j \Gamma\indices{^\ell_{ik}} + \Gamma\indices{^m_{jk}}\Gamma\indices{^\ell_{im}} - \Gamma\indices{^m_{ik}}\Gamma\indices{^\ell_{jm}},
    \end{equation*}
    onde \(\Gamma\indices{^k_{ij}}\) são os coeficientes da conexão em coordenadas locais.
\end{proposition}
\begin{proof}
    Seja \((U, x) \in \mathscr{A}_M\) uma carta de coordenadas locais, então
    \begin{equation*}
        \left[\bfield{x^i}, \bfield{x^j}\right] = 0
    \end{equation*}
    para quaisquer \(i,j\). Dessa forma, temos
    \begin{align*}
        R\indices{^\ell_{kij}} &= \inner*{\dl{x}^\ell}{R\left(\bfield{x^i}, \bfield{x^j}\right)\bfield{x^k}}\\
                               &= \inner*{\dl{x}^\ell}{\nabla_{\bfield{x^i}} \nabla_{\bfield{x^j}} \bfield{x^k} - \nabla_{\bfield{x^j}} \nabla_{\bfield{x^i}} \bfield{x^k} - \nabla_{\left[\bfield{x^i}, \bfield{x^j}\right]}\bfield{x^k}}\\
                               &= \inner*{\dl{x}^\ell}{\nabla_{\bfield{x^i}}\left(\Gamma\indices{^m_{jk}}\bfield{x^m}\right) - \nabla_{\bfield{x^j}}\left(\Gamma\indices{^m_{ik}}\bfield{x^m}\right)}\\
                               &= \inner*{\dl{x}^\ell}{\partial_i \Gamma\indices{^m_{jk}}\bfield{x^m} + \Gamma\indices{^m_{jk}} \nabla_{\bfield{x^i}}\bfield{x^m} - \partial_j \Gamma\indices{^m_{ik}}\bfield{x^m} - \Gamma\indices{^m_{ik}}\nabla_{\bfield{x^j}}\bfield{x^m}}\\
                               &= \inner*{\dl{x}^\ell}{\partial_i \Gamma\indices{^m_{jk}}\bfield{x^m} + \Gamma\indices{^m_{jk}} \Gamma\indices{^n_{im}} \bfield{x^n} - \partial_j \Gamma\indices{^m_{ik}}\bfield{x^m} - \Gamma\indices{^m_{ik}} \Gamma\indices{^n_{jm}} \bfield{x^n}}\\
                               &= \partial_i \Gamma\indices{^\ell_{jk}} + \Gamma\indices{^m_{jk}}\Gamma\indices{^\ell_{im}} - \partial_j \Gamma\indices{^\ell_{ik}} - \Gamma\indices{^m_{ik}}\Gamma\indices{^\ell_{jm}},
    \end{align*}
    como queríamos mostrar.
\end{proof}

\begin{lemma}{Identidade de Palatini}{palatini}
    Seja \((M, g)\) uma variedade Lorentziana dotada de uma conexão livre de torção \(\nabla\), cujos coeficientes em uma carta local de coordenadas são \(\Gamma\indices{^k_{ij}}\). Uma variação \(\delta \Gamma\indices{^k_{ij}}\) nos coeficientes da conexão resulta na variação do tensor de curvatura de Riemann \(\delta R^{\ell_{kij}}\) e no tensor de Ricci \(\delta R_{kj}\) dadas por
    \begin{equation*}
        \delta R\indices{^\ell_{kij}} = \nabla_i \delta \Gamma\indices{^\ell_{jk}} - \nabla_j \delta \Gamma\indices{^\ell_{ik}}
        \quad\text{e}\quad
        \delta R_{kj} = \nabla_i \delta \Gamma\indices{^i_{jk}} - \nabla_j \delta \Gamma\indices{^i_{ik}}.
    \end{equation*}
\end{lemma}
\begin{proof}
    Consideremos o tensor de curvatura de Riemann, com expressão em coordenadas locais dadas por
    \begin{equation*}
        R\indices{^\ell_{kij}} = \partial_i \Gamma\indices{^\ell_{jk}} - \partial_j \Gamma\indices{^\ell_{ik}} + \Gamma\indices{^m_{jk}}\Gamma\indices{^\ell_{im}} - \Gamma\indices{^m_{ik}}\Gamma\indices{^\ell_{jm}},
    \end{equation*}
    então uma transformação \(\Gamma\indices{^k_{ij}} \to  \Gamma\indices{^k_{ij}} + \delta \Gamma{^k_{ij}}\) resulta na transformação \(R\indices{^\ell_{kij}} \to R\indices{^\ell_{kij}} + \delta R\indices{^\ell_{kij}}\), onde
    \begin{equation*}
        \delta R\indices{^\ell_{kij}} = \partial_i \delta\Gamma\indices{^\ell_{jk}} - \partial_j \delta\Gamma\indices{^\ell_{ik}} + \Gamma\indices{^m_{jk}}\delta\Gamma\indices{^\ell_{im}} + \Gamma\indices{^\ell_{im}}\delta\Gamma\indices{^m_{jk}} - \Gamma\indices{^m_{ik}}\delta\Gamma\indices{^\ell_{jm}} - \Gamma\indices{^\ell_{jm}}\delta\Gamma\indices{^m_{ik}}.
    \end{equation*}
    Pelo \cref{lem:conexões_tensor}, segue que \(\delta \Gamma\indices{^k_{ij}}\) é um tensor, portanto sua derivada covariante está bem definida e é dada em coordenadas locais por
    \begin{equation*}
        \nabla_{i} \delta\Gamma\indices{^\ell_{jk}} = \partial_i \delta \Gamma\indices{^\ell_{jk}} + \Gamma\indices{^\ell_{im}} \delta\Gamma\indices{^m_{jk}} - \Gamma\indices{^m_{ij}} \delta\Gamma\indices{^\ell_{mk}} - \Gamma\indices{^m_{ik}}\delta\Gamma\indices{^\ell_{jm}}.
    \end{equation*}
    Podemos escrever a variação dos componentes do tensor de curvatura como
    \begin{align*}
        \delta R\indices{^\ell_{kij}} &= \left(\partial_i \delta\Gamma\indices{^\ell_{jk}}+ \Gamma\indices{^\ell_{im}}\delta\Gamma\indices{^m_{jk}} - \Gamma\indices{^m_{ik}}\delta\Gamma\indices{^\ell_{jm}} \right) - \left(\partial_j \delta\Gamma\indices{^\ell_{ik}} + \Gamma\indices{^\ell_{jm}}\delta\Gamma\indices{^m_{ik}}- \Gamma\indices{^m_{jk}}\delta\Gamma\indices{^\ell_{im}} \right)\\
                                      &= \left(\nabla_i \delta\Gamma\indices{^\ell_{jk}} + \Gamma\indices{^m_{ij}} \delta\Gamma\indices{^\ell_{mk}}\right) - \left(\nabla_j \delta\Gamma\indices{^\ell_{ik}} + \Gamma\indices{^m_{ji}}\delta\Gamma\indices{^\ell_{mk}}\right)\\
                                      &= \nabla_i \delta\Gamma\indices{^\ell_{jk}} - \nabla_j \delta\Gamma\indices{^\ell_{ik}},
    \end{align*}
    já que a conexão é simétrica.
\end{proof}

\begin{theorem}{Variação de Palatini}{palatini}
    Seja \((M,g)\) uma variedade Lorentziana dotada de uma conexão livre de torção \(\nabla\), cujos coeficientes em uma carta local de coordenadas são \(\Gamma\indices{^k_{ij}}\). As equações de Euler-Lagrange para a ação
    \begin{equation*}
        S[g_{ij}, \Gamma\indices{^k_{ij}}] = \int_M \sqrt{-g}\dln4x \left(\frac{1}{2 \kappa} R + \mathcal{L}_\mathrm{m}\right),
    \end{equation*}
    em que \(R = g^{ij}R_{ij}\) é o escalar de Ricci com \(R_{ij}\) o tensor de Ricci, \(\kappa\) é uma constante, e \(\mathcal{L}_\mathrm{m}\) é uma densidade de lagrangiana para a matéria que não depende da conexão, são dadas por
    \begin{equation*}
        R_{ij} - \frac12 Rg_{ij} = \kappa T_{ij}\quad\text{e}\quad \Gamma\indices{^k_{ij}} = \frac12 g^{k\ell}\left(\partial_ig_{\ell j} + \partial_j g_{\ell i} - \partial_\ell g_{ij}\right),
    \end{equation*}
    onde
    \begin{equation*}
        T_{ij} = \frac{-2}{\sqrt{-g}}\frac{\delta (\sqrt{-g} \mathcal{L}_\mathrm{m})}{\delta g^{ij}}
    \end{equation*}
    é o tensor de energia e momento.
\end{theorem}
\begin{proof}
    Pela variação da métrica, temos
    \begin{align*}
        \delta S &= \int_M \dln4x \left[\frac1{2 \kappa}\left(R \frac{\delta\sqrt{-g}}{\delta g^{ij}} + \sqrt{-g} \frac{\delta R}{\delta g^{ij}}\right) + \frac{\delta (\sqrt{-g} \mathcal{L}_\mathrm{m})}{\delta g^{ij}}\right] \delta g^{ij}\\
                 &=\int_M \sqrt{-g}\dln4x \left[\frac1{2 \kappa}\left(\frac{R}{\sqrt{-g}} \frac{\delta\sqrt{-g}}{\delta g^{ij}} +  \frac{\delta R}{\delta g^{ij}}\right) + \frac{1}{\sqrt{-g}}\frac{\delta (\sqrt{-g} \mathcal{L}_\mathrm{m})}{\delta g^{ij}}\right] \delta g^{ij},
    \end{align*}
    isto é, a derivada funcional da ação em relação à métrica é
    \begin{align*}
        \frac{\delta S}{\delta g^{ij}} &= \frac{1}{2 \kappa}\left(\frac{R}{\sqrt{-g}} \frac{\delta \sqrt{-g}}{\delta g^{ij}} + \frac{\delta R}{\delta g^{ij}} + \frac{2 \kappa}{\sqrt{-g}}\frac{\delta(\sqrt{-g}\mathcal{L}_\mathrm{m})}{\delta g^{ij}}\right)\\
                                       &=\frac{1}{2 \kappa}\left(\frac{R}{2g} \frac{\delta g}{\delta g^{ij}} + R_{ij} - \kappa T_{ij}\right),
    \end{align*}
    em que utilizamos a independência do tensor de Ricci com a métrica. Pela fórmula de Jacobi, temos
    \begin{equation*}
        \delta g = g g^{ij} \delta g_{ij},
    \end{equation*}
    portanto, como \(g^{ij} \delta g_{ij} + g_{ij} \delta g^{ij} = 0\), temos
    \begin{equation*}
        \frac{\delta S}{\delta g^{ij}} = \frac{1}{2 \kappa}\left(R_{ij} - \frac{1}{2}R g_{ij} - \kappa T_{ij}\right).
    \end{equation*}

    Pela variação dos coeficientes da conexão, temos
    \begin{equation*}
        \delta S = \frac{1}{2 \kappa}\int_M \sqrt{-g}\dln4x g^{ij} \left(\nabla_m \delta \Gamma\indices{^m_{ji}} - \nabla_j \delta \Gamma\indices{^m_{mi}}\right),
    \end{equation*}
    pelo \cref{lem:palatini} e pela densidade de lagrangiana da matéria ser independente da conexão. Consideremos a conexão de Levi-Civita \(\tilde{\nabla}\), então pelo \cref{lem:conexões_tensor}, existe um tensor \(D\) tal que para todos \(X, Y \in \sections{TM}\)
    \begin{equation*}
        \nabla_X Y = \tilde{\nabla}_X Y + D(X,Y),
    \end{equation*}
    de modo que
    \begin{align*}
        2 \kappa \delta S &= \int_M\sqrt{-g}\dln4x g^{ij} \begin{multlined}[t]\left[\left(\tilde{\nabla}_m \delta\Gamma\indices{^m_{ji}}+ D\indices{^m_{mn}} \delta\Gamma\indices{^n_{ji}} - D\indices{^n_{mj}} \delta\Gamma\indices{^m_{ni}} - D\indices{^n_{mi}}\delta\Gamma\indices{^m_{jn}}\right)\right. \\ - \left.\left(\tilde{\nabla}_j \delta\Gamma\indices{^m_{mi}}+ D\indices{^m_{jn}} \delta\Gamma\indices{^n_{mi}} - D\indices{^n_{jm}} \delta\Gamma\indices{^m_{ni}} - D\indices{^n_{ji}}\delta\Gamma\indices{^m_{mn}}\right)\right]\end{multlined}\\
                          &= \int_M\sqrt{-g}\dln4x \begin{multlined}[t]\left[
                              \tilde{\nabla}_m\left(g^{ij}\delta\Gamma\indices{^m_{ji}} - g^{im}\delta\Gamma\indices{^n_{ni}}\right) + g^{ij}\left(D\indices{^n_{jm}} - D\indices{^n_{mj}}\right)\delta\Gamma\indices{^m_{ni}}\right.\\\left.+ g^{ij}\left(D\indices{^m_{mn}}\delta\Gamma\indices{^n_{ji}} - D\indices{^n_{mi}}\delta\Gamma\indices{^m_{jn}} - D\indices{^m_{jn}}\delta\Gamma\indices{^n_{mi}} + D\indices{^n_{ji}}\delta\Gamma\indices{^m_{mn}}\right)\right].
                          \end{multlined}
    \end{align*}
    O termo dado por uma derivada covariante em relação à métrica de Levi-Civita resulta em um termo de fronteira e o termo dada pela antissimetrização do tensor simétrico \(D\) se anula, portanto a variação da ação em relação aos coeficientes da conexão só é dado pelo último termo, isto é,
    \begin{align*}
        \delta S &= \frac{1}{2 \kappa}\int_M \sqrt{-g} \dln4x g^{ij}\left(D\indices{^m_{mn}}\delta\Gamma\indices{^n_{ji}} - D\indices{^n_{mi}}\delta\Gamma\indices{^m_{jn}} - D\indices{^m_{jn}}\delta\Gamma\indices{^n_{mi}} + D\indices{^n_{ji}}\delta\Gamma\indices{^m_{mn}}\right)\\
                 &= \frac{1}{2 \kappa}\int_M\sqrt{-g}\dln4x g^{ij} \left(D\indices{^\ell_{\ell k}}\delta\indices{^r_j}\delta\indices{^s_i} - D\indices{^\ell_{ki}}\delta\indices{^r_j}\delta\indices{^s_\ell} - D\indices{^\ell_{jk}}\delta\indices{^r_\ell}\delta\indices{^s_i} + D\indices{^\ell_{ji}}\delta\indices{^r_k}\delta\indices{^s_\ell}\right) \delta \Gamma\indices{^k_{rs}}\\
                 &= \frac{1}{2 \kappa}\int_M\sqrt{-g}\dln4x \left(D\indices{^\ell_{\ell k}} g^{sr} - D\indices{^s_{ki}}g^{ir} - D\indices{^r_{jk}}g^{sj} + D\indices{^s_{ji}}\delta\indices{^r_k}g^{ij}\right) \delta\Gamma\indices{^k_{rs}}.
    \end{align*}
    Assim, a derivada funcional da ação em relação aos coeficientes da conexão é dada por
    \begin{align*}
        \frac{\delta S}{\delta \Gamma\indices{^k_{ij}}} &= \frac1{2 \kappa} \left(D\indices{^\ell_{\ell k}}g^{ji} - D\indices{^j_{k\ell}}g^{\ell i} - D\indices{^i_{\ell k}}g^{j\ell} + D\indices{^j_{m \ell}}\delta\indices{^i_k} g^{\ell m}\right)\\
                                                        &= \frac{1}{2 \kappa} \left(D\indices{^\ell_{\ell k}}g^{ij} - D\indices{^{ji}_k} - D\indices{^{ij}_k} + D\indices{^{j\ell}_{\ell}}\delta\indices{^i_k}\right),
    \end{align*}
    utilizando a simetria da métrica e do tensor \(D\).

    A partir de \(\delta S = 0\) e da independência da variação em relação a métrica e à conexão, obtemos as equações de movimento
    \begin{equation*}
        R_{ij} - \frac12 Rg_{ij} = \kappa T_{ij}\quad\text{e}\quad
    D\indices{^\ell_{\ell k}}g^{ij} - D\indices{^{ji}_k} - D\indices{^{ij}_k} + D\indices{^{j\ell}_{\ell}}\delta\indices{^i_k} = 0.
    \end{equation*}
    Para completar a demonstração, devemos mostrar que \(D\) é o tensor nulo, isto é, que a conexão \(\nabla\) deve ser igual à conexão de Levi-Civita. Contraímos os índices \(i\) e \(k\) na equação para \(D\), obtendo
    \begin{equation*}
        D\indices{^\ell_\ell^j} +3D\indices{^{j\ell}_\ell} - D\indices{^{\ell j}_\ell} = 0.
    \end{equation*}
    Da simetria do tensor, temos
    \begin{equation*}
        D\indices{^\ell_{\ell}^j} = g^{ij}D\indices{^\ell_{\ell i}} = g^{ij} D\indices{^\ell_{i\ell}} = D\indices{^{\ell j}_\ell},
    \end{equation*}
    portanto \(D\indices{^{j\ell}_\ell} = 0\).
    Contraímos a equação de movimento para \(D\) com \(g_{ij}\), obtendo
    \begin{equation*}
        2D\indices{^\ell_{\ell k}} + D\indices{_k^\ell_\ell} = 0,
    \end{equation*}
    portanto, como \(D\indices{_k^\ell_\ell} = g_{kj}D\indices{^{j\ell}_\ell}\), segue que \(D\indices{^\ell_{\ell k}} = 0\). Substituindo na equação de movimento, temos
    \begin{equation*}
        D\indices{^{ij}_k} + D\indices{^{ji}_k} = 0
    \end{equation*}
    portanto contraindo com \(g^{kn}\) e tomando permutações cíclicas de  \(\set{i,j,n}\), temos
    \begin{equation*}
        \begin{cases}
            D^{ijn} + D^{jin} = 0\\
            D^{jni} + D^{nji} = 0\\
            D^{nij} + D^{inj} = 0
        \end{cases}.
    \end{equation*}
    Somando as duas primeiras equações e subtraindo a terceira, temos
    \begin{equation*}
        \left(D^{ijn} - D^{inj}\right) + \left(D^{jin} + D^{jni}\right) + \left(D^{nji} - D^{nij}\right) = 0,
    \end{equation*}
    portanto pela simetria do último par de índices, temos \(D^{jin} = 0\), isto é, \(D\) é o tensor nulo.
\end{proof}
\end{document}
