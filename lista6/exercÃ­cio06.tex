\section*{Exercício 6}
Para a métrica de Friedmann-Lemaître-Robertson-Walker,
\begin{equation*}
    \dl[2]{s} = -\dl[2]{t} + a^2(t) \gamma_{ij} \dl{x}^i\dl{x}^j\text{, com }\gamma_{ij} = \frac{\delta_{ij}}{\left[1 + \frac{\kappa}4(x^2 + y^2 + z^2)\right]^2},
\end{equation*}
onde \(\kappa\) é a curvatura da seção espacial, o escalar de Ricci é dado por
\begin{equation*}
    R = 6\left(\frac{\ddot{a}}{a} + \frac{\dot{a}^2}{a^2} + \frac{\kappa}{a^2}\right).
\end{equation*}
Para um fator de escala com comportamento do tipo \(a(t) = a_0t^n\), temos
\begin{equation*}
    R = 6 \left(\frac{n(n-1)}{t^2} + \frac{n^2}{t^2} + \frac{\kappa}{a_0^2t^{2n}}\right) \implies \lim_{t \to 0^+} \abs{R} = \infty,
\end{equation*}
isto é, \(t \to 0^+\) é uma singularidade em que a curvatura do espaço-tempo é divergente, não importando o tipo de curvatura da seção espacial. Diferentemente da singularidade na origem de uma métrica de Schwarzschild, esta singularidade na métrica de FLRW não possui um horizonte de eventos, uma vez que ela está bem definida para todo \(t > 0\).
