\section*{Exercício 9}
Notemos que a derivada direcional \(W^\mu \partial_\mu \phi\) para um campo escalar \(\phi\) na direção dada pelo campo vetorial \(W\) é um escalar. De fato, em relação à outra carta de coordenadas temos
\begin{equation*}
    W^\mu \partial_\mu \phi = \left(\diffp{x^\mu}{x'^\alpha}W'^\alpha\right)\left(\diffp{x'^\beta}{x^\mu}\partial'_\beta\right)\phi = W'^\alpha \partial'_\alpha \phi,
\end{equation*}
uma vez que o campo escalar é independente da escolha de cartas.

Desse modo, com o resultado do exercício anterior, segue que a integral
\begin{equation*}
    \int_M \dl.dn.[D]{x}\sqrt{g} W^\mu\partial_\mu \phi
\end{equation*}
é invariante. Suponhamos agora que o suporte de \(\phi\) é contido no domínio de uma carta \(x\), então por integração por partes temos
\begin{align*}
    \int_M \dl.dn.[D]{x}\sqrt{g} W^\mu\partial_\mu\phi &= \int_{\partial M} \dl.dn.[D-1]{x}\sqrt{g} n_\mu W^\mu \phi - \int_M \dl.dn.[D]{x}\phi \partial_\mu\left(\sqrt{g} W^\mu\right)\\
                                                       &= \int_{\partial M} \dl.dn.[D-1]{x}\sqrt{g} n_\mu W^\mu \phi - \int_M \dl.dn.[D]{x}\sqrt{g}\phi \frac{\partial_\mu\left(\sqrt{g} W^\mu\right)}{\sqrt{g}}
\end{align*}
onde \(n_\mu\) é a 1-forma definida pela fronteira \(\partial M\). Equivalentemente temos
\begin{equation*}
    \int_{\partial M}\dl.dn.[D-1]{x} \sqrt{g} n_\mu W^\mu \phi = \int_M \dl.dn.[D]{x}\sqrt{g}\left[W^\mu \partial_\mu \phi + \phi \frac{\partial_\mu(\sqrt{g}W^\mu)}{\sqrt{g}}\right],
\end{equation*}
então pelo teorema do divergente,
\begin{equation*}
    \int_{\partial M}\dl.dn.[D-1]{x} \sqrt{g} n_\mu W^\mu \phi = \int_M \dl.dn.[D]{x}\sqrt{g} \mathrm{div}(\phi W),
\end{equation*}
segue que
\begin{equation*}
    \mathrm{div}(\phi W) = W^\mu \partial_\mu \phi + \phi \frac{\partial_\mu(\sqrt{g}W^\mu)}{\sqrt{g}}.
\end{equation*}

Pela propriedade do divergente
\begin{equation*}
    \mathrm{div}(\phi W) = W^\mu\partial_\mu\phi + \phi\mathrm{div}(W) = W^\mu\partial_\mu\phi +\phi \mathrm{div}(W),
\end{equation*}
obtemos
\begin{equation*}
    \colorunderline{Mauve}{W^\mu \partial_\mu \phi} + \phi \mathrm{div}(W)= \colorunderline{Mauve}{W^\mu \partial_\mu \phi} + \phi \frac{\partial_\mu (\sqrt{g} W^\mu)}{\sqrt{g}} \implies \phi \left[ \mathrm{div}(W) - \frac{\partial_\mu(\sqrt{g}W^\mu)}{\sqrt{g}}\right] = 0.
\end{equation*}
Como o campo escalar é arbitrário, temos
\begin{equation*}
    \mathrm{div}(W) = \frac{\partial_\mu(\sqrt{g} W^\mu)}{\sqrt{g}}.
\end{equation*}

Notemos que
\begin{equation*}
    \mathrm{div}(W) = \partial_\mu W^\mu + \frac{\partial_\nu\sqrt{g}}{\sqrt{g}}W^\nu.
\end{equation*}
Nas coordenadas normais de Riemann, temos \(\tilde{g} = 1\) e \(\tilde{\partial}_\mu \tilde{g} = 0\), portanto
\begin{equation*}
    \mathrm{div}(W) = \tilde{\partial}_\mu\tilde{W}^\mu.
\end{equation*}
Assim, em uma carta de coordenadas arbitrária temos
\begin{equation*}
    \mathrm{div}(W) = \nabla_\mu W^\mu.
\end{equation*}
Comparando \(\nabla_\mu W^\mu = \partial_\mu W^\mu + \Gamma\indices{^\mu_{\mu\nu}}W^\nu\) com a expressão para o divergente, temos
\begin{equation*}
    \Gamma\indices{^\mu_{\mu\nu}} = \frac{\partial_\nu \sqrt{g}}{\sqrt{g}}.
\end{equation*}
Substituindo \(W^\mu\) pelo gradiente de um campo escalar \((\mathrm{grad} \psi)^\mu = g^{\mu\nu}\partial_\nu \psi\), obtemos a expressão para o seu laplaciano, dado por
\begin{equation*}
    \nabla_\mu \nabla^\mu\psi = \frac{1}{\sqrt{g}}\partial_\mu\left(\sqrt{g} g^{\mu\nu}\partial_\nu \psi\right).
\end{equation*}

Podemos utilizar estas identidades para encontrar as expressões para o divergente e laplaciano em coordenadas esféricas para o espaço Euclidiano,
\begin{equation*}
    \dl[2]{s} = \dl[2]{r} + r^2\dl[2]\theta + r^2 \sin^2\theta\dl[2]\varphi,
\end{equation*}
onde a métrica tem determinante dado por \(g = r^4 \sin^2\theta.\)
Para um campo de vetores \(W = W^r \vetor{e}_r + rW^\theta \frac{\vetor{e}_\theta}{r} + r\sin\theta W^\varphi \vetor{e}_\varphi\), temos
\begin{align*}
    \mathrm{div}(W) &= \frac1{r^2 \sin\theta}\partial_\mu\left(r^2\sin\theta W^\mu\right)\\
                    &= \frac{1}{r^2\sin\theta}\partial_r\left(r^2 \sin\theta W^r\right) +\frac{1}{r^2\sin\theta}\partial_\theta\left(r^2 \sin\theta W^\theta\right) +\frac{1}{r^2\sin\theta}\partial_\varphi\left(r^2 \sin\theta W^\varphi\right)\\
                    &= \frac{1}{r^2}\partial_r \left(r^2 W^r\right) + \frac{1}{\sin\theta}\partial_\theta\left(\sin\theta W^\theta\right) + \partial_\varphi W^\varphi
\end{align*}
