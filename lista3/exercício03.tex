\section*{Exercício 3}
Consideremos coordenadas esféricas para o espaço tridimensional Euclidiano, dadas por
\begin{equation*}
    r = \sqrt{x^2 + y^2 + z^2},\quad \cos\theta = \frac{z}{r},\quad\text{e}\quad\tan\phi=\frac{y}{x}.
\end{equation*}
Alternativamente, temos
\begin{equation*}
    x = r\sin\theta\cos\phi,\quad y = r\sin\theta\sin\phi,\quad\text{e}\quad z = r\cos\theta,
\end{equation*}
de modo que os vetores da base no sistema de coordenadas esféricas são dados por
\begin{align*}
    \vetor{e}_r &= \diffp{x}{r}\vetor{e}_x + \diffp{y}{r}\vetor{e}_y + \diffp{z}{r}\vetor{e}_z\\
                &= \sin\theta\cos\phi\vetor{e}_x + \sin\theta\sin\phi\vetor{e}_y + \cos\theta \vetor{e}_z,
\end{align*}
\begin{align*}
    \vetor{e}_\theta &= \diffp{x}{\theta}\vetor{e}_x + \diffp{y}{\theta}\vetor{e}_y + \diffp{z}{\theta}\vetor{e}_z \\
                   &= r\cos\theta\cos\phi \vetor{e}_x + r\cos\theta\sin\phi\vetor{e}_y - r\sin\theta\vetor{e}_z,
\end{align*}
e
\begin{align*}
    \vetor{e}_\phi &= \diffp{x}{\phi}\vetor{e}_x + \diffp{y}{\phi}\vetor{e}_y + \diffp{z}{\phi}\vetor{e}_z\\
                   &= -r\sin\theta\sin\phi \vetor{e}_x + r\sin\theta\cos\phi\vetor{e}_y.
\end{align*}

Assim, os coeficientes da métrica Euclidiana nas coordenadas esféricas são dados por
\begin{equation*}
    g_{rr} = 1, \quad g_{\theta\theta} = r^2, \quad\text{e}\quad g_{\phi\phi} = r^2\sin^2\theta,
\end{equation*}
e as outras componentes nulas.

Consideremos agora a métrica da relatividade restrita \(\eta_{\mu\nu}\) e as coordenadas em rotação
\begin{equation*}
    \begin{cases}
        t' = t\\
        x' = \sqrt{x^2 + y^2}\cos{(\phi - \omega t)}\\
        y' = \sqrt{x^2 + y^2}\sin{(\phi - \omega t)}\\
        z' = z
    \end{cases}
\end{equation*}
onde \(\tan\phi = \frac{y}{x}\). Notemos que
\begin{equation*}
    x' = x\cos{\omega t} + y\sin{\omega t}\quad\text{e}\quad y' = -x\sin{\omega t} + y\cos{\omega t}
\end{equation*}
então ao tomar combinações lineares das equaççoes
