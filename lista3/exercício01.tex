\section*{Exercício 1}
No contexto da mecânica Newtoniana, a massa inercial \(m_i\) de uma partícula é relacionada à força resultante que age nela pela segunda lei de Newton, \(\vetor{F} = m_i\vetor{a}\). Com sua lei de gravitação, temos que a força gravitacional é dada por \(\vetor{F}_g = -m_g\nabla\Phi\), onde \(m_g\) é a massa gravitacional e \(\Phi\) é o potencial gravitacional. O Princípio de Equivalência Fraco diz que a massa inercial inercial e a massa gravitacional são iguais, de modo que qualquer partícula em queda livre tem aceleração dada por \(\vetor{a} = -\nabla \Phi\). A série de experimentos por Eötvös no fim do século XIX verificou o Princípio de Equivalência Fraco com precisão de \(5\times10^{-9}\), enquanto que atualmente a precisão é da ordem de \(10^{-15}\).

Ainda, em uma região suficientemente pequena, podemos aproximar o gradiente \(-\nabla \Phi\) para uma constante \(\vetor{g}\), de modo que nesta região todas as partículas em queda livre têm aceleração uniforme igual a \(\vetor{g}\). Assim, um campo gravitacional homogêneo é equivalente à uma aceleração do sistema de referência. O Princípio de Equivalência de Einstein diz que, em regiões suficientemente pequenas do espaço-tempo, vale a Relatividade Restrita e e que é impossível detectar a existência de um campo gravitacional por experimentos locais. Isto é, localmente um campo gravitacional é indistinguível à um referencial uniformemente acelerado, ilustrado pelo \textit{Gedankenexperiment} do elevador de Einstein.

O Princípio de Equivalência Forte diz que para uma trajetória de uma partícula massiva em queda livre em um campo gravitacional qualquer, é possível escolher um sistema de coordenadas localmente inercial, de modo que, em uma região do espaço-tempo suficientemente pequena ao redor desta trajetória, todas as leis físicas são equivalentes às suas formulações em sistemas de referência não acelerados na ausência da gravidade.
