\section*{Exercício 8}
% Consideremos a transformação da métrica e da 1-forma \(\dl{x}^\sigma\) entre cartas locais \(x\) e \(x'\),
% \begin{equation*}
%     g_{\mu\nu} = \diffp{x'^\alpha}{x^\mu}\diffp{x'^\beta}{x^\nu} g'_{\alpha\beta}\quad\text{e}\quad\dl{x}^\sigma = \diffp{x^\sigma}{x'^\rho}\dl{x'}^\rho.
% \end{equation*}
Consideremos a \(D\)-forma \(\dl{x}^0 \wedge \dots \wedge \dl{x}^{D-1}\), então em outra carta de coordenadas \(x'\), temos
\begin{align*}
    \dl{x}^0 \wedge \dots \wedge \dl{x}^{D-1} &= \left(\diffp{x^0}{x'^{\mu_0}} \dl{x'}^{\mu_0}\right) \wedge \dots \wedge \left(\diffp{x^{D-1}}{x'^{\mu_{D-1}}}\dl{x'}^{\mu_{D-1}}\right)\\
                                              &= \epsilon^{\mu_0\dots\mu_{D-1}} \left(\diffp{x^0}{x'^\mu_0}\dots\diffp{x^{D-1}}{x'^{\mu_{D-1}}}\right) \dl{x'}^0\wedge\dots \wedge\dl{x'}^{D-1}\\
                                              &= J \dl{x'}^0\wedge\dots \wedge\dl{x'}^{D-1},
\end{align*}
onde \(J\) é o determinante do jacobiano \(\diffp{x^\alpha}{x'^\beta}\), isto é, \(J = \epsilon^{\mu_0\dots\mu_{D-1}}\diffp{x^0}{x'^{\mu_0}} \dots \diffp{x^{D-1}}{x'^{\mu_{D-1}}}\).

Consideremos agora a transformação da métrica \(g_{\mu\nu}\) das coordenadas \(x\) para as coordenadas \(x'\)
\begin{equation*}
    g_{\mu\nu} = \diffp{x'^\alpha}{x^\mu}\diffp{x'^\beta}{x^\nu}g'_{\alpha\beta}.
\end{equation*}
Notemos que podemos arranjar esta última equação como uma multiplicação matricial
\begin{equation*}
    (g_{\mu\nu}) = \left(\diffp{x'^\alpha}{x^\mu}\right)^\top(g'_{\alpha\beta})\left(\diffp{x'^\beta}{x^\nu}\right),
\end{equation*}
donde segue que
\begin{equation*}
    g = J^{-2} g',
\end{equation*}
onde \(g\) e \(g'\) são os valores absolutos dos determinantes das matrizes que representam a métrica nas cartas de coordenadas locais \(x\) e \(x'\). Desse modo, o objeto \(\sqrt{g}\) se transforma de forma inversa à \(D\)-forma considerada. Nesse caso, obtemos a regra de transformação
\begin{equation*}
    \sqrt{g} \dl{x}^0 \wedge \dots \wedge \dl{x}^{D-1} = \sqrt{g'} \dl{x'}^0 \wedge \dots \wedge \dl{x'}^{D-1},
\end{equation*}
isto é,\(\sqrt{g} \dl{x}^0 \wedge \dots \wedge{x}^{D-1}\) se transforma como um escalar, então podemos utilizar este objeto como uma medida invariante para integrais em uma variedade.

Por exemplo, consideremos o círculo unitário submerso no plano Euclidiano \(S^1 \subset \mathbb{R}^2\) e a carta
\begin{align*}
    \psi : S^1 \smallsetminus \set{(1,0)} \subset \mathbb{R}^2 &\to (0, 2\pi)\\
    (\cos\theta,\sin\theta)&\mapsto \theta.
\end{align*}
Então o vetor da base induzida por esta carta é \(\vetor{e}_\theta = (-\sin\theta, \cos\theta)\), que é unitário em relação à métrica Euclidiana. Isto é, a métrica induzida em \(S^1\) é dada por seu único elemento \(g_{\theta\theta} = 1\), e cujo determinante é \(g = 1\). Desse modo, o elemento de linha para o círculo unitário é \(\dl{s} = \sqrt{g}\dl{\theta} = \dl{\theta},\) como esperado.

Com a medida invariante podemos também determinar volumes invariantes \todo[preciso entender o enunciado melhor.]
