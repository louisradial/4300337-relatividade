\section*{Exercício 6}
Um espaço topológico é uma dupla \topology{M} composta por um conjunto \(M\) e uma topologia \(\mathcal{O}_M\). Um subconjunto \(U\) de \(M\) é dito ser aberto em relação a este espaço topológico se \(U \in \mathcal{O}_M\). Uma aplicação \(f : M \to N\) entre espaços topológicos \topology{M} e \topology{N} é dita contínua se sua pré-imagem de um aberto é aberta, e é dita um homeomorfismo se for bijetiva e tanto \(f\) quanto \(f^{-1}\) forem contínuas. Se existe um homeomorfismo entre dois espaços topológicos, estes são ditos homeomorfos.

Se existe um número inteiro \(n\) tal que todo aberto \(U \in \mathcal{O}_M\) é homeomorfo a \(\mathbb{R}^n\), em relação à topologia usual do espaço Euclidiano, dizemos que \topology{M} é um espaço topológico localmente Euclidiano de dimensão \(n\). Ainda, para cada aberto \(U \in \mathcal{O}_M\) existe um homeomorfismo \(x : U \to x(U) \subset \mathbb{R}^n\), e chamamos o par \((U,x)\) de carta local. Um atlas \(\mathscr{A}_M\) é uma coleção de cartas locais tal que a união dos abertos cobre o conjunto \(M\).

Consideremos agora duas cartas \((U, x), (V, x) \in \mathscr{A}_M\) tal que \(U \cap V \neq \emptyset\).
\begin{equation*}
    \begin{tikzcd}[column sep = normal, row sep = large]
        &\arrow[swap]{dl}{x} U \cap V\arrow{dr}{y}&\\
        x(U\cap V) \arrow{rr}{y \circ x^{-1}} & & y(U\cap V)
    \end{tikzcd}
\end{equation*}
Como uma composição de homeomorfismos, segue que a aplicação de transição \(y \circ x^{-1} : \mathbb{R}^n \to \mathbb{R}^n\) é um homeomorfismo, isto é, contínua. Como uma função em \(\mathbb{R}^n\), podemos utilizar análise usual para decidir se esta função é diferenciável. Duas cartas locais \((U, x), (V, y)\) são ditas \(\mathcal{C}^k\)-compatíveis se ou \(U \cap V \neq \emptyset\) e a aplicação de transição \(y \circ x^{-1}\) é de classe \(\mathcal{C}^k\) ou se \(U \cap V = \emptyset\). Ainda, um atlas é dito \(\mathcal{C}^k\)-compatível se todo par de cartas locais são \(\mathcal{C}^k\)-compatíveis.

Uma variedade diferenciável \manifold{M} é um espaço topológico \topology{M} localmente Euclidiano munido de um atlas maximal suave \(\mathscr{A}_M\), isto é, um atlas \(\mathcal{C}^\infty\)-compatível com a propriedade de que se uma carta \((U,x)\) é compatível com uma carta \((V,y) \in \mathscr{A}_M\), então \((U,x) \in \mathscr{A}_M\). A estrutura diferencial dada pelo atlas permite definir em todo ponto \(p \in M\) um espaço vetorial \(T_pM\), chamado de espaço tangente no ponto \(p\), cujos elementos são derivações na álgebra \smooth{M} de funções suaves \(f : M \to \mathbb{R}\). Geometricamente, cada elemento \(X \in T_pM\) é um operador de derivada direcional ao longo de alguma curva suave \(\gamma : (-\varepsilon, \varepsilon) \to M\) que passa por \(p = \gamma(0)\). O espaço dual \(T^{\ast}_pM\) é chamado de espaço cotangente no ponto \(p\), cujos elementos são relacionados com as curvas de nível de funções suaves \smooth{M}.

Utilizando o atlas da variedade, podemos definir um atlas para a união disjunta dos espaços tangentes, construindo assim o fibrado tangente \(TM\), que é também uma variedade diferenciável. Uma aplicação suave \(p \mapsto X_p\) que associa um ponto \(p\) da variedade a um vetor \(X_p \in T_pM \subset TM\) do fibrado tangente é chamada de campo de vetores. Analogamente, definimos o fibrado cotangente \(T ^{\ast}M\), em que uma aplicação suave \(p \mapsto \omega_p\) que associa um ponto \(p \in M\) a um elemento \(\omega_p \in T_p ^{\ast}M \subset T ^{\ast}M\) é chamada de 1-forma diferencial, ou campo de covetores. Uma função multilinear de campos de vetores e de 1-formas diferenciais é chamada de tensor na variedade.

Resumindo de forma mais informal, uma variedade diferenciável é um conjunto \(M\) que localmente se parece com algum espaço Euclidiano \(\mathbb{R}^n\), e no qual podemos definir ponto a ponto um espaço vetorial, que é intimamente relacionado à estrutura diferencial fornecida à \(M\) por um atlas de cartas de coordenadas locais. Um tensor no contexto de uma variedade diferenciável é uma função multilinear de vetores e 1-formas definida em todo ponto da variedade.
