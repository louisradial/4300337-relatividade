\section*{Exercício 10}
Consideremos a métrica de um espaço em expansão
\begin{equation*}
    \dl[2]s = - \dl[2]t + t^{2q}\left(\dl[2]{x} + \dl[2]{y} + \dl[2]{z}\right),
\end{equation*}
com \(q \in (0,1)\) e \(t \in (0,\infty)\). Para um intervalo tipo luz ao longo do eixo \(x\), temos
\begin{equation*}
    \dl[2]t = t^{2q}\dl[2]x \implies t^{-q} \dl{t} = \pm\dl{x} \implies t^{-q} \diff{t}{x} = \pm 1,
\end{equation*}
isto é, uma equação diferencial para a coordenada temporal. Integrando em relação à posição obtemos
\begin{equation*}
    \frac{t^{1-q}}{1-q} = \pm(x - \xi) \implies t = \left[\pm(1-q)(x - \xi)\right]^{\frac{1}{1-q}},
\end{equation*}
onde \(\xi\) é uma constante de integração.

\begin{figure}[H]
    \centering
    \includegraphics[width=0.5\textwidth]{exercício10.png}
    \caption{Diagrama de espaço-tempo.}
    \label{fig:exercício10}
\end{figure}

Pelo diagrama de espaço-tempo mostrado na \cref{fig:exercício10}, notamos que para pontos distintos não é necessário que haja interseção de seus passados.
