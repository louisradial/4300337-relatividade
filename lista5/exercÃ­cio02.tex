\section*{Exercício 2}
Como obtido no exercício anterior, o potencial efetivo é dado por
\begin{equation*}
    V_\mathrm{ef}(r) = - \kappa \frac{GM}{r} + \frac{\ell^2}{2r^2} - \frac{GM\ell^2}{r^3},
\end{equation*}
cujo possível gráfico é apresentado na \cref{fig:exercício2}, a depender das relações entre os parâmetros \(M\) e \(\ell\).

\begin{figure}[ht]
    \centering
    \begin{tikzpicture}
        \begin{axis}[
            width=0.8\linewidth,
            height=0.25\textheight,
            xmin=0, xmax=24,
            ymin=-0.1,ymax=0.6,
            domain=0.5:24,
            samples=500,
            axis lines=middle,
            xlabel={$r$},
            ylabel={$V_\mathrm{ef}$},
            legend pos=north east,
            ytick=\empty,
            xtick=\empty
        ]
            \addplot[thick, Mauve] {-1/(2*x) + 25/(8*x^2) - 25/(8*x^3)};
            \addlegendentry{\(\kappa = 1\)}

            \addplot[thick, Teal] {25/(8*x^2) - 25/(8*x^3)};
            \addlegendentry{\(\kappa = 0\)}
        \end{axis}
    \end{tikzpicture}
    \caption{Potencial efetivo radial}
    \label{fig:exercício2}
\end{figure}

Determinemos os pontos críticos do potencial efetivo. Temos
\begin{equation*}
    \diff{V_\mathrm{ef}}{r} = \frac{\kappa GM r^2 - \ell^2 r + 3 GM\ell^2}{r^4},
    %\quad\text{e}\quad\diff[2]{V_\mathrm{ef}}{r} = -\frac{2 \kappa GM  r^2 - 3\ell^2r + 12 GM \ell^2}{r^5},
\end{equation*}
portanto um ponto crítico do potencial efetivo satisfaz a equação polinomial
\begin{equation*}
    \kappa GM r^2 - \ell^2r + 3 GM\ell^2 = 0.
\end{equation*}

Para partículas massivas, temos as raízes
\begin{equation*}
    2GM r_{\pm} = \ell^2 \pm \ell^2\sqrt{1 - 3 \left(\frac{2GM}{\ell}\right)^2},
\end{equation*}
para \(\abs{\ell} \geq 2\sqrt{3} GM\) e nenhuma raiz real caso contrário. Isto é, podemos escrever
\begin{equation*}
    \diff{V_\mathrm{ef}}{r} = \frac{GM (r - r_-)(r-r_+)}{r^4},
\end{equation*}
evidenciando que
\begin{align*}
    0 < r < r_- &\implies \diff{V_\mathrm{ef}}{r} > 0& r_- < r < r_+ &\implies \diff{V_\mathrm{ef}}{r} < 0 & r_+ < r &\implies \diff{V_\mathrm{ef}}{r} > 0,
\end{align*}
isto é, se \(r_- \neq r_+\), o potencial efetivo é crescente antes de \(r_-\) e depois de \(r_+\), e decrescente entre \(r_-\) e \(r_+\). Dessa forma, como o potencial efetivo é suave para \(0 < r < \infty\), segue que \(r_-\) e \(r_+\) são pontos de máximo e de mínimo, respectivamente, a não ser que se igualem, caso em que são um ponto de inflexão. Ou seja, para \(\abs{\ell} > 2\sqrt{3}GM\) temos as órbitas circulares em
\begin{equation*}
    r_\mathrm{min} = \frac{\ell^2}{2GM}\left[1 - \sqrt{1 - 3\left(\frac{2GM}{\ell}\right)^2}\right]\quad\text{e}\quad
    r_\mathrm{max} = \frac{\ell^2}{2GM}\left[1 + \sqrt{1 - 3\left(\frac{2GM}{\ell}\right)^2}\right],
\end{equation*}
que são instável e estável, respectivamente. Sejam \(V_\mathrm{min} = V_\mathrm{ef}(r_\mathrm{max})\) e \(V_\mathrm{min} = V_\mathrm{ef}(r_\mathrm{max})\) os potenciais efetivos das órbitas circulares, então para \(V_\mathrm{min} < \frac{\epsilon^2 - 1}{2} < 0\) temos órbitas limitadas não circulares, enquanto que para \(0 \leq \frac{\epsilon^2 - 1}{2} < V_\mathrm{max}\) temos órbitas abertas, com \(r > r_\mathrm{min}\) indo ao infinito e \(r < r_\mathrm{min}\) indo à origem. Notemos que para o limite \(\ell \to 2\sqrt{3}GM\), temos
\begin{equation*}
    \lim_{\ell \to 2\sqrt{3}GM} r_\mathrm{max} = \frac{\ell^2}{2GM} = 6GM,
\end{equation*}
portanto o raio da órbita circular mais interna \(r_\mathrm{ISCO}\) é igual a três vezes o raio de Schwarzschild.

Para partículas de massa nula, temos a única raiz \(r_0 = 3GM\). Escrevemos
\begin{equation*}
    \diff{U_\mathrm{ef}}{r} = \frac{2}{\ell^2}\diff{V_\mathrm{ef}}{r} = -2\frac{r - r_0}{r^4},
\end{equation*}
portanto temos que \(U_\mathrm{ef}\) cresce em \(0 < r < r_0\) e decresce em \(r > r_0\), isto é, \(r_0\) é um ponto de máximo local. Assim, fótons podem percorrer órbitas circulares instáveis de raio igual a três meios do raio de Schwarzschild. Nesta órbita, temos \(U_\mathrm{ef}(r_0) = \frac{1}{27 G^2M^2}\), portanto o parâmetro de impacto deve ser tal que \(\frac1{b^2} = \frac{1}{27G^2M^2}\). Para parâmetros de impacto que não satisfazem esta relação, temos órbitas abertas, com \(r < r_0\) tendendo à origem e com \(r > r_0\) tendendo ao infinito.

% Substituindo essas raízes na segunda derivada do potencial efetivo, obtemos
% \begin{equation*}
%     \diff[2]{V_\mathrm{ef}}{r}[r_{\pm}] = \pm16\left(\frac{GM}{r_{\pm}}\right)^4\ell^2\sqrt{1 - 3\left(\frac{2GM}{\ell}\right)^2},
% \end{equation*}
% isto é, se \(\abs{\ell} = 2\sqrt{3}GM,\) há apenas um ponto de inflexão em \(r = \frac{\ell^2}{2GM}\) e se \(\abs{\ell} > 2\sqrt{3}GM\), há um máximo em \(r_\mathrm{max} = r_-\) e um mínimo em \(r_\mathrm{min}=r_+\).
