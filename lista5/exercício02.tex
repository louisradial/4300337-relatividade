\section*{Exercício 2}
Como obtido no exercício anterior, o potencial efetivo é dado por
\begin{equation*}
    V_\mathrm{ef}(r) = - \kappa \frac{GM}{r} + \frac{\ell^2}{2r^2} - \frac{GM\ell^2}{r^3},
\end{equation*}
cujo possível gráfico é apresentado na \cref{fig:exercício2}, a depender das relações entre os parâmetros \(M\) e \(\ell\).

\begin{figure}[ht]
    \centering
    \begin{tikzpicture}
        \begin{axis}[
            width=0.8\linewidth,
            height=0.3\textheight,
            xmin=0, xmax=24,
            ymin=-0.1,ymax=0.6,
            domain=0.5:24,
            samples=500,
            axis lines=middle,
            xlabel={$r$},
            ylabel={$V_\mathrm{ef}$},
            legend pos=north east,
            ytick=\empty,
            xtick=\empty
        ]
            \addplot[thick, Mauve] {-1/(2*x) + 25/(8*x^2) - 25/(8*x^3)};
            \addlegendentry{\(\kappa = 1\)}

            \addplot[thick, Teal] {25/(8*x^2) - 25/(8*x^3)};
            \addlegendentry{\(\kappa = 0\)}
        \end{axis}
    \end{tikzpicture}
    \caption{Potencial efetivo radial}
    \label{fig:exercício2}
\end{figure}

Determinemos os pontos críticos do potencial efetivo. Temos
\begin{equation*}
    \diff{V_\mathrm{ef}}{r} = \frac{\kappa GM r^2 - \ell^2 r + 3 GM\ell^2}{r^4}\quad\text{e}\quad\diff[2]{V_\mathrm{ef}}{r} = -\frac{2 \kappa GM  r^2 - 3\ell^2r + 12 GM \ell^2}{r^5},
\end{equation*}
portanto um ponto crítico do potencial efetivo satisfaz a equação polinomial
\begin{equation*}
    \kappa GM r^2 - \ell^2r + 3 GM\ell^2 = 0.
\end{equation*}
Para partículas massivas, temos as raízes
\begin{equation*}
    2GM r_{\pm} = \ell^2 \pm \ell^2\sqrt{1 - 3 \left(\frac{2GM}{\ell}\right)^2},
\end{equation*}
para \(\abs{\ell} \geq 2\sqrt{3} GM\) e nenhuma raiz real caso contrário. Substituindo essas raízes na segunda derivada do potencial efetivo, obtemos
\begin{equation*}
    \diff[2]{V_\mathrm{ef}}{r}[r_{\pm}] = \pm16\left(\frac{GM}{r_{\pm}}\right)^4\ell^2\sqrt{1 - 3\left(\frac{2GM}{\ell}\right)^2},
\end{equation*}
isto é, se \(\abs{\ell} = 2\sqrt{3}GM,\) há apenas um ponto de inflexão em \(r = \frac{\ell^2}{2GM}\) e se \(\abs{\ell} > 2\sqrt{3}GM\), há um máximo em \(r_\mathrm{max} = r_-\) e um mínimo em \(r_\mathrm{min}=r_+\).
