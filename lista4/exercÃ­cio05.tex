\section*{Exercício 5}

Consideremos a ação
\begin{equation*}
    S_\mathrm{M} = \int_M \dln4x\sqrt{-g}\mathcal{L}_\mathrm{M},
\end{equation*}
para uma densidade de lagrangiana \(\mathcal{L}_\mathrm{M}\) que depende da métrica.
Definindo o tensor de energia e momento por
\begin{equation*}
    T_{\mu\nu} = \frac{-2}{\sqrt{-g}}\diffp{(\sqrt{-g}\mathcal{L}_\mathrm{M})}{g^{\mu\nu}},
\end{equation*}
a variação da ação para uma variação do tensor métrico \(g^{\mu\nu} \to g^{\mu\nu} + \delta g^{\mu\nu}\) é dada por
\begin{equation*}
    \delta S_\mathrm{M} = \int_M \dln4x \diffp{(\sqrt{-g}\mathcal{L}_\mathrm{M})}{g^{\mu\nu}}\delta g^{\mu\nu} =-\frac12\int_M\dln4x \sqrt{-g}T_{\mu\nu} \delta g^{\mu\nu}.
\end{equation*}
% em que o possível termo em relação às derivadas de \(g^{\mu\nu}\) foi ignorado por estarmos considerando teorias com uma conexão de Levi-Civita, isto é, a lagrangiana \(\mathcal{L}_\mathrm{M}\) não pode depender da conexão.

Consideremos o caso particular do eletromagnetismo, com a ação dada por
\begin{equation*}
    S_\mathrm{EM} = \int_M\dln4x \sqrt{-g}\left(-\frac14 g^{\rho \alpha}g^{\sigma \beta}F_{\rho\sigma}F_{\alpha \beta}\right),
\end{equation*}
onde \(F_{\mu\nu}\) é o tensor de Faraday. Assim, variando a métrica e lembrando que o tensor de Faraday é antissimétrico, temos
\begin{align*}
    \delta S_\mathrm{EM} &= -\frac14\int_{M} \dln4x \sqrt{-g}\delta\left(g^{\rho \alpha}g^{\sigma \beta}F_{\rho\sigma}F_{\alpha\beta}\right) - \frac14 \int_{M} \dln4x F_{\rho\sigma}F^{\rho\sigma} \delta \sqrt{-g}\\
                         &= - \frac14 \int_{M} \dln4x \sqrt{-g} \left(\delta\indices{^\rho_\mu}\delta\indices{^\alpha_\nu} g^{\sigma\beta} + g^{\rho\alpha}\delta\indices{^\sigma_\mu}\delta\indices{^\beta_\nu}\right)F_{\rho\sigma}F_{\alpha\beta}\delta g^{\mu\nu} + \frac18\int_{M} \dln4x \sqrt{-g}g_{\mu\nu}F_{\rho\sigma}F^{\rho\sigma}\delta g^{\mu\nu}\\
                         &= - \frac12 \int_{M} \dln4x \sqrt{-g} \left[\frac12 \left(g^{\sigma\beta}F_{\mu\sigma}F_{\nu\beta} + g^{\rho\alpha}F_{\rho\mu}F_{\alpha\nu}\right) - \frac14 g_{\mu\nu}F_{\rho\sigma}F^{\rho\sigma}\right]\delta g^{\mu\nu}\\
                         &= - \frac12 \int_{M} \dln4x \sqrt{-g} \left[\frac12 \left( g^{\kappa\lambda}F_{\mu\kappa}F_{\nu\lambda} + g^{\kappa\lambda}F_{\kappa\mu}F_{\lambda\nu}\right) - \frac14 g_{\mu\nu}F_{\rho\sigma}F^{\rho\sigma}\right]\delta g^{\mu\nu}\\
                         &= - \frac12 \int_{M} \dln4x \sqrt{-g} \left(g^{\kappa\lambda}F_{\kappa\mu}F_{\lambda\nu} - \frac14 g_{\mu\nu}F_{\rho\sigma}F^{\rho\sigma}\right)\delta g^{\mu\nu},
\end{align*}
então o tensor de energia e momento é
\begin{equation*}
    T_{\mu\nu} = F\indices{^\kappa_\nu}F_{\kappa\mu} - \frac14 g_{\mu\nu} F_{\rho\sigma}F^{\rho\sigma},
\end{equation*}
ou então, com índices contravariantes,
\begin{align*}
    T^{\mu\nu} = g^{\mu\alpha}g^{\nu\beta}T_{\alpha\beta} &= g^{\mu\alpha}g^{\nu\beta} \left(F\indices{^\kappa_\beta}F_{\kappa\alpha} - \frac14 g_{\alpha\beta} F_{\rho\sigma}F^{\rho\sigma}\right)\\
                                                          &= F\indices{^\mu_\kappa}F^{\nu\kappa} - \frac14g^{\mu\nu}F_{\rho\sigma}F^{\rho\sigma}.
\end{align*}

Consideremos as coordenadas normais de Riemann, em que podemos tomar os resultados conhecidos da Relatividade Restrita. Isto é, nesta carta temos
\begin{equation*}
    F^{0i} = E^{i}\quad\text{e}\quad F^{ij} = \epsilon^{ijk}B_k,
\end{equation*}
com as demais componentes dadas pela antissimetria do tensor. A fim de calcular as componentes \(T^{00}\) e \(T^{i0}\) do tensor de energia e momento, determinamos
\begin{align*}
    F_{\rho \sigma} F^{\rho\sigma} &= F_{0\sigma}F^{0\sigma} + F_{i\sigma}F^{i\sigma}&
    F\indices{^0_\kappa}F\indices{^{0\kappa}} &= F\indices{^0_k}F\indices{^{0k}}&
    F\indices{^i_\kappa}F\indices{^{0\kappa}} &= F\indices{^i_k}F^{0k}\\
                                              &= F_{0i}F^{0i} + F_{i0}F^{i0} + F_{ij}F^{ij}&
                                              &= \eta_{jk}F^{0j}E^k&
                                              &= \eta_{jk}F^{ij}E^k\\
                                              &= -2E_i E^i + \epsilon_{ij\ell}B^\ell\epsilon^{ijk}B_k&
                                              &= E_kE^k&
                                              &= \epsilon^{ij\ell}E_jB_\ell\\
                                              &= -2\norm{\vetor{E}}^2 + 2\norm{\vetor{B}}^2,&
                                              &= \norm{\vetor{E}}^2,&
                                              &= (\vetor{E}\times\vetor{B})^i.
\end{align*}
Assim, obtemos as componentes desejadas,
\begin{equation*}
    T^{00} = \norm{\vetor{E}}^2 + \frac{2\norm{\vetor{B}}^2 - 2\norm{\vetor{E}}}4 = \frac{\norm{\vetor{E}}^2 + \norm{\vetor{B}}^2}{2}\quad\text{e}\quad T^{i0} = (\vetor{E}\times\vetor{B})^i,
\end{equation*}
isto é, \(T^{00}\) é a densidade de energia do campo eletromagnético e \(T^{i0}\) é a componente \(i\) do vetor de Poynting.
