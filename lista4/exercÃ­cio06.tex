\section*{Exercício 6}
No exercício 1, mostramos que vale
\begin{equation*}
    R_{\lambda\mu\rho\nu} = \frac12 R \left(g_{\lambda\rho} g_{\mu\nu} - g_{\lambda\nu}g_{\mu\rho}\right)
\end{equation*}
no caso particular de variedades bidimensionais. Assim, o tensor de Riemann é dado por
\begin{equation*}
    R\indices{^\sigma_{\mu\rho\nu}} = g^{\sigma\lambda} R_{\lambda\mu\rho\nu} = \frac12 R g^{\sigma\lambda}\left(g_{\lambda\rho} g_{\mu\nu} - g_{\lambda\nu}g_{\mu\rho}\right) = \frac12 R\left(\delta\indices{^\sigma_\rho}g_{\mu\nu} - \delta\indices{^\sigma_\nu}g_{\mu\rho}\right),
\end{equation*}
de modo que o tensor de Ricci é obtido pela contração de \(\sigma\) e \(\rho\),
\begin{equation*}
    R_{\mu\nu} = R\indices{^\sigma_{\mu\sigma\nu}} = \frac12 R\left(2g_{\mu\nu} - g_{\mu\nu}\right) = \frac12 Rg_{\mu\nu}.
\end{equation*}
Dessa forma, o tensor de Einstein em duas dimensões é identicamente nulo,
\begin{equation*}
    G_{\mu\nu} = R_{\mu\nu} - \frac12 Rg_{\mu\nu} = 0.
\end{equation*}
Dessa forma, as equações de Einstein \(G_{\mu\nu} = \kappa T_{\mu\nu}\), implicam que o tensor de energia e momento é nulo em toda a variedade.
