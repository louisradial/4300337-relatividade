\section*{Exercício 3}
O grupo de Lorentz \(\mathrm{O}(1,3)\) pode ser representado como \begin{equation*}
    \mathrm{O}(1,3) = \set*{\Lambda \in \mathrm{GL}(\mathbb{R}^4) : \Lambda^\top \eta \Lambda = \eta},
\end{equation*}
em que \(\eta\) é a matriz dada por
\begin{equation*}
    \eta = \begin{pmatrix}
        -1 & 0 & 0 & 0\\
        0 & 1 & 0 & 0\\
        0 & 0 & 1 & 0\\
        0 & 0 & 0 & 1\\
    \end{pmatrix},
\end{equation*}
representando a métrica de Minkowski.

Notemos que se duas matrizes são iguais, então certamente os seus determinantes também o são. Desse modo, para \(\Lambda \in \mathrm{O}(1,3)\) temos
\begin{align*}
    \Lambda^\top \eta \Lambda = \eta &\implies \det\left(\Lambda^\top \eta \Lambda\right) = \det \eta\\
                                     &\implies \det \Lambda^\top \det \eta \det \Lambda = \det \eta\\
                                     &\implies \left(\det \Lambda\right)^2 = 1\\
                                     &\implies \det \Lambda = \pm 1,
\end{align*}
já que o determinante da matriz transposta é igual ao determinante da matriz.

Em termos das componentes, temos
\begin{equation*}
    \eta_{\mu\nu} = \eta_{\rho\sigma}\Lambda\indices{^\rho_\mu}\Lambda\indices{^\sigma_\nu},
\end{equation*}
utilizando a notação de Einstein. Em particular, para \(\mu = \nu = 0\),
\begin{equation*}
    \eta_{\rho\sigma}\Lambda\indices{^\rho_0}\Lambda\indices{^\sigma_0} = \eta_{00},
\end{equation*}
ou de forma mais explícita,
\begin{equation*}
    \left(\Lambda\indices{^0_0}\right)^2 = 1 + \left(\Lambda\indices{^1_0}\right)^2+ \left(\Lambda\indices{^2_0}\right)^2+ \left(\Lambda\indices{^3_0}\right)^2.
\end{equation*}
Assim, como os elementos de \(\Lambda\) são números reais, devemos ter \(\left|\Lambda\indices{^0_0}\right| \geq 1\).

Notemos que uma reflexão dos eixos espaciais como \((ct, x,y,z) \mapsto (ct, -x, y, z)\) é uma transformação ortogonal em relação à métrica de Minkowski, já que uma reflexão dos eixos em \(\mathbb{R}^3\) é uma transformação ortogonal em relação à métrica Euclidiana. O determinante de transformações deste tipo é sempre igual a \(-1\). Semelhantemente, transformações que refletem o eixo temporal \((ct, x,y, z) \mapsto ((-ct, x,y, z))\) também é ortogonal em relação à métrica de Minkowski e tem determinante \(-1\). Deste modo, para ignorar transformações deste tipo, devemos adicionar a restrição \(\det \Lambda = 1\). Definimos o grupo
\begin{equation*}
    \mathrm{SO}(1,3) = \set*{\Lambda \in \mathrm{O}(1,3) : \det \Lambda = 1}
\end{equation*}
das transformações ortogonais em relação à métrica de Minkowski, exceto as reflexões espaciais e temporais.

Entretanto, uma composição de uma reflexão espacial e de uma reflexão temporal tem determinante unitário. Nestes casos, a componente \(\Lambda\indices{^0_0}\) deve ser negativa, portanto podemos restringir o grupo de Lorentz para conter apenas \textit{boosts} e rotações com o grupo
\begin{equation*}
    \mathrm{SO}^\uparrow(1,3) = \set*{\Lambda \in \mathrm{SO}(1,3) : \Lambda\indices{^0_0} \geq 1},
\end{equation*}
chamado de grupo de Lorentz restrito.

Mostremos que o conjunto \(\mathrm{SO}(1,3)\) é um grupo sob composição de transformações lineares, isto é, sob multiplicação matricial. Notemos que a identidade \(\id{\mathbb{R}^4} \in \mathrm{GL}(\mathbb{R}^4)\) pertence a \(\mathrm{SO}^\uparrow(1,3) \subset \mathrm{SO}(1,3)\), já que \(\det\id{\mathbb{R}^4} = 1\) e \({\id{\mathbb{R}^4}}\indices{^0_0} = 1\). Como um subconjunto do grupo \(\mathrm{GL}(\mathbb{R}^4)\), a composição de transformações de Lorentz é certamente associativa, portanto devemos mostrar que esta composição é também um elemento de \(\mathrm{SO}^\uparrow(1,3)\). De fato, sejam \(\Lambda, M \in \mathrm{SO}(1,3)\), então para \(N = \Lambda \cdot M\) temos
\begin{align*}
    N^\top \cdot \eta \cdot N &= (\Lambda \cdot M)^\top \cdot \eta \cdot (\Lambda \cdot M)\\
                                  &= M^\top \cdot \Lambda^\top \cdot \eta \cdot \Lambda \cdot M\\
                                  &= M^\top \cdot \eta \cdot M\\
                                  &= \eta,
\end{align*}
logo \(N \in \mathrm{O}(1,3)\);
\begin{equation*}
    \det N = \det \Lambda \det M = 1,
\end{equation*}
logo \(N \in \mathrm{SO}(1,3)\); vale notar que caso \(\Lambda, M \in \mathrm{SO}^\uparrow(1,3)\), então é possível (mas não tão direto) mostrar que \(N \in \mathrm{SO}^\uparrow(1,3)\). Resta mostrar que para todo \(\Lambda \in \mathrm{SO}(1,3)\) temos \(\Lambda^{-1} \in \mathrm{SO}(1,3)\). De \(\Lambda \in \mathrm{O}(1,3)\), temos
\begin{equation*}
    \Lambda^\top\cdot \eta\cdot \Lambda = \eta \implies \Lambda^{-1} = \eta \cdot \Lambda^\top \cdot \eta,
\end{equation*}
então é claro que \(\det \left(\Lambda^{-1}\right) = 1\), e
\begin{align*}
    \left(\Lambda^{-1}\right)^\top \cdot \eta \cdot \Lambda^{-1} &= \left(\eta \cdot \Lambda \cdot \eta \right) \cdot \eta \cdot \left(\eta \cdot \Lambda^\top \cdot \eta\right)\\
                                                                 &= \eta \cdot \left(\Lambda^\top \cdot \eta \cdot \Lambda \right)^\top \cdot \eta\\
                                                                 &= \eta,
\end{align*}
isto é, \(\Lambda^{-1} \in \mathrm{SO}(1,3)\). Deste modo, mostramos que \(\mathrm{SO}(1,3)\) é um grupo. Relaxando a condição do determinante unitário, mostramos com o mesmo argumento que \(\mathrm{O}(1,3)\) é um grupo.

