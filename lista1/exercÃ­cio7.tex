\section*{Exercício 7}
No referencial \(\Sigma'\) de repouso da barra, suas extremidades se encontram em todo instante no plano \(x_0y_0\) na origem e no ponto \((L_0 \cos\theta_0, L_0 \sin \theta_0)\).

O referencial \(\Sigma'\) se move em relação ao referencial \(\Sigma\) com velocidade \(v\hat{x}\). Pela contração de Lorentz, as extremidades da barra se encontram nas posições \(\left(vt,0\right)\) e \(\left(vt + \frac{L_0\cos\theta_0}{\gamma}, L_0\sin\theta_0\right)\) em um dado instante \(t\). Assim, o comprimento da barra neste referencial é
\begin{equation*}
    L = \sqrt{L_0^2\sin^2\theta_0 + \frac{L_0^2\cos^2\theta_0}{\gamma^2}} = \frac{L_0}{\gamma} \sqrt{\gamma^2\sin^2\theta_0 + \cos^2\theta_0}
\end{equation*}
e o ângulo \(\theta\) que a barra faz com o eixo \(x\) é dado por
\begin{equation*}
    \tan \theta = \gamma \tan \theta_0.
\end{equation*}

