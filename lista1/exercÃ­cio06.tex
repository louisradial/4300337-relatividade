\section*{Exercício 6}
Seja \(\Sigma\) o referencial em que \(A\) se move com velocidade \(v_A^C=\frac45c\) e \(B\) se move com velocidade \(v_B^C = \frac35c\) na mesma direção. Neste referencial, a velocidade relativa entre \(A\) e \(B\) é \(v = \frac15c\), de modo que o tempo que \(A\) leva para ultrapassar \(B\) é
\begin{equation*}
    \Delta t = \frac{L_A + L_B}{v},
\end{equation*}
em que \(L_A\) é o comprimento de \(A\) e \(L_B\) de \(B\) neste referencial. Assim,
\begin{equation*}
    \Delta t = \frac{\frac{1}{\gamma_A} + \frac{1}{\gamma_B}}{v}L = \frac{7L}{c}.
\end{equation*}
