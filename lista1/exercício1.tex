\section*{Exercício 1}
No referencial \(K\), a partícula se move com velocidade \(v = 0.998c\) em direção ao chão. Assim, se a produção do múon ocorre à altitude \(h \approx \SI{15}{\kilo\meter}\), então o tempo \(t\) transcorrido desde o tempo de produção até a chegada da partícula no chão é
\begin{equation*}
    t = \frac{h}{v} \approx \SI{5.0e-5}{\second}.
\end{equation*}
Se \(\tau' = \SI{2.2e-6}{\second}\) é a vida média no referencial \(K'\) de repouso do múon, então no referencial \(K\), a vida média é
\begin{equation*}
    \tau = \gamma(v)\tau' = \SI{3.5e-5}{\second},
\end{equation*}
em que \(\gamma(v) = \left[1 - \left(\frac{v}{c}\right)^2\right]^{-\frac12}\).
Assim, a probabilidade da detecção de um múon no chão neste referencial é
\begin{equation*}
    p = \exp{\left(-\frac{t}{\tau}\right)} \approx 0.24.
\end{equation*}

No referencial \(K'\), o chão se move em direção à partícula com velocidade \(v\). Pela contração de Lorentz, se a distância percorrida no referencial \(K\) é \(h\), no referencial \(K'\) o chão se move uma distância \(h'\) dada por
\begin{equation*}
    h' = \frac{h}{\gamma(v)} \approx \SI{0.95}{\kilo\meter}.
\end{equation*}
Dessa forma, o tempo \(t'\) transcorrido desde a produção do múon e a chegada do chão ao múon é
\begin{equation*}
    t' = \frac{h'}{v} \approx \SI{3.2e-6}{\second}.
\end{equation*}
Assim, a probabilidade da detecção de um múon no chão neste referencial é
\begin{equation*}
    p' = \exp{\left(-\frac{t'}{\tau'}\right)} \approx 0.24,
\end{equation*}
o mesmo valor obtido no referencial \(K\).
