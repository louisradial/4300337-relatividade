\section*{Exercício 3}
O processo de propulsão certamente deve respeitar a conservação de energia e momento, neste caso, temos
\begin{equation*}
    d(\gamma mv) = -dp\quad\text{e}\quad d(\gamma mc^2) = c dp,
\end{equation*}
onde \(p\) é o momento do fóton. Desta forma, temos
\begin{align*}
    d\left(\gamma mc + \gamma m v\right) = 0.
\end{align*}
Notemos que
\begin{equation*}
    d \gamma = \gamma^3 \frac{v}{c^2} dv,
\end{equation*}
portanto
\begin{equation*}
    \gamma (v + c) dm + m \gamma \left(\gamma^2\frac{v(v + c)}{c^2} + 1\right) dv = 0.
\end{equation*}
Dividindo por \(\gamma m (v + c),\) obtemos
\begin{equation*}
    \frac{dm}{m} + \frac{\gamma^2}{c}dv = 0.
\end{equation*}
Em termos de \(\beta = \frac{v}{c}\), temos
\begin{equation*}
    \frac{dm}{m} + \frac{d\beta}{1 - \beta^2} = 0.
\end{equation*}

Se o foguete parte do repouso com massa inicial \(M\), temos
\begin{align*}
    \ln{\left(\frac{m}{M}\right)} &= -\int_0^\beta \frac{\dl\xi}{1 - \xi^2}\\
                                  &= -\frac12 \int_0^\beta \left(\frac1{\xi+1} - \frac1{\xi-1}\right)\dl\xi\\
                                  &= -\frac12\ln{\left(\frac{\beta + 1}{1 - \beta}\right)}\\
                                  &= \ln{\sqrt{\frac{1 - \beta}{1+ \beta}}}.
\end{align*}
Desse modo, o foguete tem massa
\begin{equation*}
    m = M\sqrt{\frac{1 - \beta}{1 + \beta}}
\end{equation*}
quando sua velocidade é igual a \(\beta c\).

Se \(\sigma = \diff{m}{\tau}\) é a taxa em que o foguete converte massa em fótons no referencial instantaneamente de repouso do foguete, então a taxa no referencial terreste é \(\frac\sigma\gamma = \diff{m}{t}\). Neste caso, temos
\begin{align*}
    \frac{\sigma}{M} &= \gamma  \diff*{\sqrt{\frac{1 - \beta}{1+\beta}}}{t}\\
                     &= -\gamma  \frac{\frac{1}{(1+\beta)^2}}{\sqrt{\frac{1-\beta}{1+\beta}}}\diff{\beta}{t}\\
                     &= -\frac{1}{(1+\beta)^2(1-\beta)} \diff{\beta}{t},
\end{align*}
isto é, uma equação diferencial a variáveis separáveis,
\begin{equation*}
    -\frac{\sigma}{M} \dl{t} = \frac{\dl{\beta}}{(1+\beta)^2(1-\beta)}.
\end{equation*}

Integrando, obtemos
\begin{align*}
    t &= -\frac{M}{\sigma} \int_0^\beta \frac{\dl\xi}{(1+\xi)^2(1-\xi)}\\
      &= -\frac{M}{4\sigma} \int_0^\beta \left(\frac{2}{(1+\xi)^2} + \frac{1}{1+\xi} - \frac{1}{\xi - 1}\right)\dl\xi\\
      &= \frac{M}{4 \sigma} \left[2(1+\xi)^{-1} + \ln{\left(\frac{1-\xi}{1+\xi}\right)}\right]_0^\beta\\
      &= \frac{M}{4 \sigma} \left[-\frac{2\beta}{1+\beta} + \ln{\left(\frac{1-\beta}{1+\beta}\right)}\right].
\end{align*}
