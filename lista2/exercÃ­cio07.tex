\section*{Exercício 7}

A ação para uma partícula de massa \(m\) é dada por
\begin{align*}
    S &= -\int mc\sqrt{-\eta_{\mu\nu} \dl{x^\mu} \dl{x^\nu}}\\
      &= - \int \dli{t}mc \sqrt{-\eta_{\mu\nu} \dot{x}^\mu \dot{x}^\nu}\\
      &= - \int \dli{t} \frac{mc^2}{\gamma(\dot{\boldsymbol{x}})},
\end{align*}
portanto a lagrangiana é dada por
\begin{align*}
    \mathcal{L}(\boldsymbol{x}, \dot{\boldsymbol{x}}) &= - mc\sqrt{-\eta_{\mu\nu}\dot{x}^\mu\dot{x}^\nu}\\
                                        &= - \frac{mc^2}{\gamma(\dot{\boldsymbol{x}})}.
\end{align*}

Deste modo, o momento canônico é dado por
\begin{align*}
    p_i = \diffp{\mathcal{L}}{\dot{x}^i} &= -mc\diffp*{\sqrt{-\eta_{\mu\nu}\dot{x}^\mu\dot{x}^\nu}}{\dot{x}^i}\\
                                         &= \frac{mc}{2\sqrt{-\eta_{\mu\nu}\dot{x}^\mu \dot{x}^\nu}}\diffp*{\eta_{\mu\nu}\dot{x}^\mu\dot{x}^\nu}{\dot{x}^i}\\
                                         &= \frac{mc}{\sqrt{-\eta_{\mu\nu}\dot{x}^\mu\dot{x}^\nu}} \eta_{\mu\nu} \delta\indices{^\mu_i} \dot{x}^\nu\\
                                         &= m\gamma(\dot{\boldsymbol{x}}) \delta^\mu_i \dot{x}_\mu\\
                                         &= m \gamma(\dot{\boldsymbol{x}}) \dot{x}_i,
\end{align*}
e a Hamiltoniana por
\begin{align*}
    \mathcal{H} = p_i\dot{x}^i - \mathcal{L} &= m \gamma \dot{x}_i\dot{x}^i + \frac{mc^2}{\gamma}\\
                                             &=  \gamma mc^2\left(\frac{\dot{x}_i\dot{x}^i}{c^2} + \frac{1}{\gamma^2}\right)\\
                                             &= \gamma mc^2,
\end{align*}
uma vez que \(\frac{1}{\gamma^2} = 1 - \frac{\dot{x}_i\dot{x}^i}{c^2}.\)

Consideremos a ação
\begin{equation*}
    \tilde{S} = -\frac12 \int \dli{\zeta}\left[\sigma(\zeta)\left(\diff{x}{\zeta}\right)^2 + \frac{m^2}{\sigma(\zeta)}\right]
\end{equation*}
então pela equação de Euler-Lagrange para \(\sigma\), temos
\begin{align*}
    \diffp*{\left[\sigma\left(\diff{x}{\zeta}\right)^2 + \frac{m^2}{\sigma}\right]}{\sigma} = 0 &\implies \left(\diff{x}{\zeta}\right)^2 - \frac{m^2}{\sigma^2} = 0\\
                                                                                                &\implies \frac{m^2}{\sigma^2} = \left(\diff{x}{\zeta}\right)^2.
\end{align*}
Assim, substituindo na ação, obtemos
\begin{equation*}
    \tilde{S} = -\int\dli{\zeta} \sigma(\zeta) \left(\diff{x}{\zeta}\right)^2.
\end{equation*}
