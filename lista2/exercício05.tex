\section*{Exercício 5}
A transformação de Lorentz de das componentes de um quadrivetor é dada por
\begin{equation*}
    x'^{\nu} = \Lambda\indices{^{\nu}_\mu} x^\mu\quad\text{e}\quad x'_{\nu} = \Lambda\indices{_{\nu}^{\mu}}x_{\mu},
\end{equation*}
então temos que as componentes \(T\indices{^{\mu_1,\dots,\mu_p}_{\nu_1,\dots,\nu_q}}\) de um tensor tipo \((p,q)\) se transformam de acordo com
\begin{equation*}
    {T'}\indices{^{\alpha_1,\dots,\alpha_p}_{\beta_1,\dots,\beta_q}} = \Lambda\indices{^{\alpha_1}_{\mu_1}}\dotsm\Lambda\indices{^{\alpha_p}_{\mu_p}} \Lambda\indices{^{\nu_1}_{\beta_1}}\dotsm\Lambda\indices{^{\nu_q}_{\beta_q}} T\indices{^{\mu_1,\dots,\mu_p}_{\nu_1,\dots,\nu_q}}.
\end{equation*}

Dadas bases \(\hat{e}_{\mu}\) e \(\hat{e}'_\nu\), temos
\begin{equation*}
    x = x^\mu \hat{e}_{\mu} = x'^{\nu} \hat{e}'_{\nu},
\end{equation*}
isto é, uma mudança de referencial não muda o vetor em si, apenas os valores das componentes. Assim,
\begin{equation*}
    x = \left(\Lambda\indices{^\nu_\sigma}x^\sigma\right)\left(\Lambda\indices{_\nu^\rho} \hat{e}_{\rho}\right) \implies \Lambda\indices{^\nu_\sigma}\Lambda\indices{_\nu^\rho} = \delta\indices{^\rho_\sigma}.
\end{equation*}

Podemos mostrar a invariância do intervalo \(\dl{s^2} = \eta_{\alpha \beta}\dl{x^{\alpha}}\dl{x^{\beta}}\). Pela transformações de tensores, temos
\begin{align*}
    \eta'_{\mu\nu} \dl{x'}^\mu \dl{x'}^\nu &= \left(\Lambda\indices{_\mu^\alpha}\Lambda\indices{_\nu^\beta}\eta_{\alpha \beta}\right)\left(\Lambda\indices{^\mu_\sigma}\dl{x}^\sigma\right)\left(\Lambda\indices{^\nu_\rho}\dl{x}^\rho\right)\\
                                           &= \delta\indices{^\alpha_\sigma}\delta\indices{^\beta_\rho} \eta_{\alpha \beta} \dl{x}^\sigma \dl{x}^\rho\\
                                           &= \eta_{\alpha \beta} \dl{x}^\alpha \dl{x}^\beta\\
                                           &= \dl{s}^2.
\end{align*}

Consideremos \(\delta\indices{^\mu_\nu} = \eta^{\mu \sigma}\eta_{\sigma \nu}\), então
\begin{align*}
    {\delta'}\indices{^\alpha_\beta} &= \Lambda\indices{^\alpha_\mu}\Lambda\indices{_\beta^\nu}\delta{^\mu_\nu}\\
                                     &= \Lambda\indices{^\alpha_\nu}\Lambda\indices{_\beta^\nu}\\
                                     &= \delta\indices{^\alpha_\beta},
\end{align*}
isto é, o delta de Kronecker é invariante por transformações de Lorentz.

Consideremos o símbolo de Levi-Civita \(\epsilon^{\mu\nu\rho\sigma}\), então
\begin{align*}
    \epsilon'^{\alpha \beta \kappa \lambda} &= \Lambda\indices{^\alpha_\mu}\Lambda\indices{^\beta_\nu}\Lambda\indices{^\kappa_\rho}\Lambda{^\lambda_\sigma} \epsilon^{\mu\nu\rho\sigma}\\
                                            &= \det{(\Lambda)} \epsilon^{\alpha \beta \kappa \lambda}.
\end{align*}
Assim, o símbolo de Levi-Civita é invariante por transformações do grupo restrito de Lorentz.
