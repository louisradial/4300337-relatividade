\section*{Exercício 4}
Consideremos a força \(F^i = \diff{p^i}{t}\). Em notação vetorial temos \(\boldsymbol{p} = \gamma m \boldsymbol{v}\), donde segue
\begin{align*}
    \boldsymbol{F} &= \diff{\gamma}{t} m \boldsymbol{v} + \gamma m \diff{\boldsymbol{v}}{t}\\
            &= \diff{\gamma}{t}  m \boldsymbol{v} + \gamma m\boldsymbol{a},
\end{align*}
onde \(\boldsymbol{a} = \diff{\boldsymbol{v}}{t}\) é a 3-aceleração da partícula.

Notemos que
\begin{align*}
    \diff{\gamma}{t} &= \left[1 - \left(\frac{v}{c}\right)^2\right]^{-\frac32} \frac{v}{c^2}\diff{v}{t}\\
                     &= \frac{\gamma^3}{c^2} \inner{\boldsymbol{v}}{\boldsymbol{a}},
\end{align*}
onde foi utilizada a relação
\begin{equation*}
    2v\diff{v}{t} = \diff*{\inner{\boldsymbol{v}}{\boldsymbol{v}}}{t} = 2\inner{\boldsymbol{v}}{\boldsymbol{a}}.
\end{equation*}
Dessa forma, obtemos
\begin{equation*}
    \boldsymbol{F} = \gamma m \left(\boldsymbol{a} + \frac{\gamma^2}{c^2}\inner{\boldsymbol{v}}{\boldsymbol{a}}\boldsymbol{v}\right).
\end{equation*}

No caso particular em que a força é paralela à velocidade, devemos ter que a aceleração é também paralela à velocidade. Assim, \(\boldsymbol{v} = v \boldsymbol{\hat{n}}\), \(\boldsymbol{a} = a\boldsymbol{\hat{n}}\) e \(\boldsymbol{F} = F\boldsymbol{\hat{n}}\) para algum vetor unitário \(\boldsymbol{\hat{n}}\), então
\begin{align*}
    F &= \gamma ma \left(1 + \gamma^2\frac{v^2}{c^2}\right)\\
      &= \gamma^3 ma.
\end{align*}

Como um exemplo, tomemos uma partícula de carga \(q\) com velocidade \(\boldsymbol{v} = v\boldsymbol{\hat{\imath}}\) que parte do repouso na origem num campo uniforme \(\boldsymbol{E} = E\boldsymbol{\hat{\imath}}\). Assim,
\begin{equation*}
    qE = \gamma^3 ma.
\end{equation*}
Notemos que isso é uma equação diferencial a variáveis separáveis
\begin{equation*}
    \left[1 - \left(\frac{v}{c}\right)^2\right]^{-\frac32}\dl{v} = \frac{qE}{m} \dl{t} \implies \frac{v}{\sqrt{1 - \left(\frac{v}{c}^2\right)}} = \frac{qEt}{m}.
\end{equation*}
Resolvendo para \(v\), obtemos
\begin{align*}
    v^2 = \frac{1}{\left(\frac{m}{qEt}\right)^2 + \left(\frac{1}{c}\right)^2} &\implies v = \sqrt{\frac{c^2}{\left(\frac{mc}{qEt}\right)^2 + 1}}\\
                                                                              &\implies v = \frac{\frac{qEt}{mc}}{\sqrt{1 + \left(\frac{qEt}{mc}\right)^2}}c.
\end{align*}
Integrando em relação ao tempo para obter a posição, temos
\begin{align*}
    x = c\int_{0}^{t}\dli{t}{\frac{\frac{qEt}{mc}}{\sqrt{1 + \left(\frac{qEt}{mc}\right)^2}}} &\implies x = \frac{mc^2}{2qE} \int_{1}^{1+\left(\frac{qEt}{mc}\right)^2} \dli{\zeta} \zeta^{-\frac12}\\
                                                                                              &\implies x = \frac{mc^2}{qE} \left[\sqrt\zeta\right]_{1}^{1+\left(\frac{qEt}{mc}\right)^2}\\
                                                                                              &\implies x = \frac{mc^2}{qE}\left(\sqrt{1 + \left(\frac{qEt}{mc}\right)^2} - 1\right),
\end{align*}
partindo de \(x = 0\).

No caso particular em que a força é ortogonal à velocidade, devemos ter
\begin{align*}
    \inner{\boldsymbol{F}}{\boldsymbol{v}} = 0 &\implies \inner{\boldsymbol{a}}{\boldsymbol{v}}\left(1 + \gamma^2 \beta^2\right) = 0\\
                                 &\implies \inner{\boldsymbol{a}}{\boldsymbol{v}} = 0,
\end{align*}
de modo que
\begin{equation*}
    \boldsymbol{F} = \gamma m\boldsymbol{a}.
\end{equation*}
Como um exemplo, tomemos uma partícula de carga \(q\) com velocidade \(\boldsymbol{v} = v \boldsymbol{\hat{\varphi}}\) contida no plano \(xy\) que se move num campo magnético uniforme \(\boldsymbol{B} = B \boldsymbol{\hat{k}}\). Assim,
\begin{align*}
    \gamma m\boldsymbol{a} = q \boldsymbol{v} \times \boldsymbol{B} &\implies -\gamma m v \dot\varphi \boldsymbol{\hat{r}} = qvB \boldsymbol{\hat{r}}\\
                                               &\implies \dot\varphi = -\frac{qB}{m \gamma},
\end{align*}
portanto a frequência \(f\) de oscilação do movimento orbital é
\begin{equation*}
    f = \frac{qB}{2\pi m}\sqrt{1 - \left(\frac{v}{c}\right)^2}.
\end{equation*}
