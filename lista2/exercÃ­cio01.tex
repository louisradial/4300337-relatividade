\section*{Exercício 1}
Para que um corpo de massa \(m\) tenha uma energia cinética \(K\), sua velocidade \(v\) satisfaz
\begin{equation*}
    \left(\frac{1}{\sqrt{1-\left(\frac{v}{c}\right)^2}} - 1\right)mc^2 = K.
\end{equation*}
Isolando \(v\), obtemos
\begin{equation*}
    v = c\sqrt{1 - \left(\frac{1}{\frac{K}{mc^2}+1}\right)^2}.
\end{equation*}

Assim, para que uma partícula tenha energia cinética igual a sua energia de repouso, sua velocidade é
\begin{equation*}
    v = \frac{\sqrt{3}}{2}c.
\end{equation*}
Pelo mesmo cálculo, para que uma bola de canhão de massa \(m = \SI{1}{\kilogram}\) tenha a mesma energia cinética que um próton, de massa \(m_p \approx\SI{1.673e-27}{\kilogram}\), de um raio cósmico em movimento com fator de Lorentz \(\gamma = 10^{11}\), sua velocidade deve ser
\begin{align*}
    v &= c\sqrt{1 - \left(\frac{1}{\frac{(\gamma - 1)m_pc^2}{mc^2}+1}\right)^2}\\
      &= c\sqrt{1 - \left(\frac{1}{\frac{(\gamma - 1)m_p}{m}+1}\right)^2}\\
      &\approx \SI{5.483}{\meter\per\second}.
\end{align*}
